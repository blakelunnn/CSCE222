\documentclass{article}
\usepackage{amsmath,amssymb,amsthm,latexsym,paralist}

\theoremstyle{definition}
\newtheorem{problem}{Problem}
\newtheorem*{solution}{Solution}
\newtheorem*{resources}{Resources}

\newcommand{\name}[2]{\noindent\textbf{Name: #1}\hfill \textbf{UIN: #2}}
\newcommand{\honor}{\noindent On my honor, as an Aggie, I have neither
  given nor received any unauthorized aid on any portion of the
  academic work included in this assignment. Furthermore, I have
  disclosed all resources (people, books, web sites, etc.) that have
  been used to prepare this homework. \\[2ex]
 \textbf{Electronic signature:} \underline{ \textbf{Blake Lun} } }
 
\newcommand{\checklist}{\noindent\textbf{Checklist:}
\begin{compactitem}[$\Box$] 
\item Did you type in your name and UIN? 
\item Did you disclose all resources that you have used? \\
(This includes all people, books, websites, etc.\ that you have consulted.)
\item Did you sign that you followed the Aggie Honor Code? 
\item Did you solve all problems? 
\item Did you submit the .tex and .pdf files of your homework to the correct link on Canvas? 
\end{compactitem}
}

\newcommand{\problemset}[1]{\begin{center}\textbf{Problem Set #1}\end{center}}
\newcommand{\duedate}[1]{\begin{quote}\textbf{Due dates:} Electronic
    submission of \textsl{yourLastName-yourFirstName-hw7.tex} and 
    \textsl{yourLastName-yourFirstName-hw7.pdf} files of this homework is due on
    \textbf{#1} on \texttt{https://canvas.tamu.edu}. You will see two separate links
    to turn in the .tex file and the .pdf file separately. Please do not archive or compress the files.  
    \textbf{If any of the two files are missing or unreadable, you will receive zero points for this
    homework.}\end{quote} }

\newcommand{\N}{\mathbf{N}}
\newcommand{\R}{\mathbf{R}}
\newcommand{\Z}{\mathbf{Z}}


\begin{document}
\vspace*{-20mm}
\begin{center}
{\large
CSCE 222 Discrete Structures for Computing -- Fall 2021\\[.5ex]
Hyunyoung Lee\\}
\end{center}
\problemset{7}
\duedate{Thursday, 10/28/2021 11:59 p.m.}
\name{ Blake Lun }{ 131001178 }
\begin{resources} (All people, books, articles, web pages, etc.\ that
  have been consulted when producing your answers to this homework)
\end{resources}
\honor

\bigskip

\noindent
Total 100 points.

\bigskip

\noindent
The problems are from the lecture notes posted on Perusall.

\medskip

\noindent
For each proof-by-induction (or strong induction) questions, strictly adhere 
to the format and the style we discussed in class.  Especially in the inductive step, 
carefully explain each step-by-step in your own words!  \textit{You will not receive 
full credit if you don't explain the steps.}

Grading rubrics for each (strong) induction proof are as follows: 20\% for the induction base
(clearly stating the base case and showing that the base case holds), 70\% for the
inductive step (correctly stating the induction hypothesis, clearly stating where
and how the induction hypothesis is used in the rest of the inductive step, clearly
and completely showing the derivation steps in the inductive step), and 10\% for the 
concluding remark of your proof. 

\medskip

\begin{problem} (20 points) Section 4.2, Exercise 4.12
\end{problem}
\begin{solution} 
Let P(n) be the inequality \[ \prod_{k=1}^{n} (1 + x_k) \geqslant 1 + \sum_{k=1}^{n} x_k\] where every $x_k \geqslant -1$ and all have the same sign.\\
Base Case: \\
P(n) holds for n = 1 since: \\
$1 + x_1 \geqslant 1 + x_1$ \\
\\
Induction Step: \\
Induction Hypothesis: Assume P(n) is true. Show that P(n) implies P(n + 1). \\
P(n + 1): \[ \prod_{k=1}^{n + 1} (1 + x_k) \geqslant 1 + \sum_{k=1}^{n + 1} x_k\] \\
Looking at just the left hand side: \[ \prod_{k=1}^{n + 1} (1 + x_k) = (1 + x_n_+_1)(\prod_{k=1}^{n} 1 + x_k)\]
We can then claim: \[(1 + x_n_+_1)(\prod_{k=1}^{n} 1 + x_k) \geqslant (1 + x_n_+_1)(1 + \sum_{k=1}^{n} x_k)\] by the induction hypothesis. \\
Foiling the right side out gives us: \[1 + \sum_{k=1}^{n} x_k + x_n_+_1 + (x_n_+_1)(\sum_{k=1}^{n} x_k)\] Now we combine the first two terms and relate it back to the right side of the original inequality to get:
\[1 + \sum_{k=1}^{n + 1} x_k + (x_n_+_1)(\sum_{k=1}^{n} x_k) \geqslant 1 + \sum_{k=1}^{n + 1} x_k\]
Looking at the inequality, we can see that $1 + \sum_{k=1}^{n + 1} x_k$ is present on both sides. Therefore, in order to prove the inequality to be true, we just need to prove that $(x_n_+_1)(\sum_{k=1}^{n} x_k) \geqslant 0$. \\
One of the definitions given at the start is every $x_k$ will have the same signs. Therefore, if $(x_n_+_1)$ is negative, $(\sum_{k=1}^{n} x_k)$ will also be negative making their product greater than or equal to 0. If $(x_n_+_1)$ is positive, $(\sum_{k=1}^{n} x_k)$ will also be positive, again, making their product greater than or equal to 0. Therefore, P(n + 1) holds. \\
\\
Therefore, we can conclude by induction that P(n) holds for all $n \geqslant 1$ which proves the claim.
\end{solution}

\begin{problem} (20 points) Section 4.6, Exercise 4.30
\end{problem}
\begin{solution} 
Let P(n) be $g(n) \leqslant 2^n$ where $g(n) = g_n_-_1 + g_n_-_2 + g_n_-_3$ for all integers $n \geqslant 3$ and $g(0) = 1$, $g(1) = 2$, and $g(2) = 3$. \\
\\
Base Case: \\
P(n) holds for n = 0, 1, 2 since: \\
$P(0) = g(0) \leqslant 2^0$ \\
$1 \leqslant 1$ \\
\\
$P(1) = g(1) \leqslant 2^1$ \\
$2 \leqslant 2$ \\
\\
$P(2) = g(2) \leqslant 2^2$ \\
$3 \leqslant 4$ \\
\\
Inductive Step: \\
Inductive Hypothesis: P(k) holds for $0 \leqslant k \leqslant n$ where $n \geqslant 3$. Prove P(n) implies P(n + 1) \\
P(n + 1): $g_n_+_1 \leqslant 2^n^+^1$ \\
$g_n_+_1 = g_n + g_n_-_1 + g_n_-_2$ \\
$g_n_+_1 = 2^n + 2^{n - 1} + 2^{n - 2}$ by induction hypothesis. \\
$\to 2^n + 2^{n - 1} + 2^{n - 2} \leqslant 2^{n + 1}$
\end{solution}

\begin{problem} (20 points) Section 11.1, Exercise 11.3
\end{problem}
\begin{solution} 
$\lim_{x\to\infty} \frac{f(n)}{g(n)}$ \\
$\to \lim_{x\to\infty} \frac{n^2 + 2n}{n^2}$ divide by n \\
$\to \lim_{x\to\infty} \frac{n + 2}{n}$ separate into two terms \\
$\to \lim_{x\to\infty} \frac{n}{n} + \frac{2}{n}$ separate into two limits \\
$\to \lim_{x\to\infty} \frac{n}{n}$ + $\lim_{x\to\infty} \frac{2}{n}$ solve both limits \\
$\to \lim_{x\to\infty} \frac{n}{n}$ $ = 1$ + $\lim_{x\to\infty} \frac{2}{n}$ $ = 0$ \\
$\to 1 + 0 = 1$ \\
Since $\lim_{x\to\infty} \frac{f(n)}{g(n)}$ $ = 1$, $f(n) \sim g(n)$ \\
Therefore, Ernie is right.
\end{solution}

\begin{problem} ($10 \text{ points } \times 2 = 20$ points) Section 11.2, Exercise 11.9 (a) and (b).
\end{problem}
\begin{solution} 
a. $f(n) = (-1)^n$ \\
when n is odd: $f(n) = (-1)^n = -1$ \\
when n is even: $f(2n) = (-1)^{2n} = ((-1)^2)^n = 1^n = 1$ \\
lower accumulation point: -1 \\
upper accumulation point: 1 \\
\\
b. $f(n) = 4 + (-1)^n \frac{n}{n + 10}$ \\
when n is odd: \\
$\lim_{n\to\infty} f(n) = 4 + (-1)^n \frac{n}{n + 10}$ apply L'Hopital's rule; $\frac{n}{n + 10}$ becomes $\frac{1}{1}$ which equals 1 \\
$\to \lim_{n\to\infty} f(n) = 4 + (-1)^n$ since n is odd, $(-1)^n = -1$ \\
$\to \lim_{n\to\infty} f(n) = 4 + (-1)$ \\
$\to \lim_{n\to\infty} f(n) = 3$ \\
\\
when n is even: \\
$\lim_{n\to\infty} f(n) = 4 + (-1)^n \frac{n}{n + 10}$ apply L'Hopital's rule; $\frac{n}{n + 10}$ becomes $\frac{1}{1}$ which equals 1 \\
$\to \lim_{n\to\infty} f(n) = 4 + (-1)^n$ since n is even, $(-1)^n = 1$ \\
$\to \lim_{n\to\infty} f(n) = 4 + 1$ \\
$\to \lim_{n\to\infty} f(n) = 5$ \\
\\

\end{solution}

\begin{problem} ($10 \text{ points } \times 2 = 20$ points) Section 11.3, Exercise 11.14 (ii) and (iii).
[Requirement: Use the definition of $\asymp$ involving the inequalities.]
\end{problem}
\begin{solution} 
ii. $f \asymp g \iff g \asymp f$ \\
\\
iii. $(f \asymp g) \sim (g \asymp h) \to (f \asymp h)$ \\
where $(f \asymp g)$ is n1, $(g \asymp h)$ is n2, $(f \asymp h)$ is n0 \\
\\
$= (c1 \vert g \vert \leqslant \vert f \vert \leqslant c2 \vert g \vert) \sim (c3 \vert h \vert \leqslant \vert g \vert \leqslant c4 \vert h \vert) \to (c5 \vert h \vert \leqslant \vert f \vert \leqslant c6 \vert h \vert)$ \\
Divide the inequalities into their own segments \\
$c1\vert g \vert \leqslant \vert f \vert \\
\vert f \vert \leqslant c2 \vert g \vert \\
c3 \vert h \vert \leqslant \vert g \vert \\
\vert g \vert \leqslant c4 \vert h \vert \\$
Using these above inequalities, we will prove \\
$c5 \vert h \vert \leqslant \vert f \vert \\
\vert f \vert \leqslant c6 \vert h \vert \\$ and find n0 \\
\\
To prove $c5 \vert h \vert \leqslant \vert f \vert$, multiply $c3 \vert h \vert \leqslant \vert g \vert$ by c1 to get $c1c3 \vert h \vert \leqslant c1\vert g \vert$ \\
c1c3, the product of these two arbitrary numbers is another arbitrary number. Therefore, we can say c1c3 = c5. Substitute this into the inequality and it gives us $c5 \vert h \vert \leqslant c1\vert g \vert$. Using the other segmented inequality of $c1\vert g \vert \leqslant \vert f \vert$, we have $c5 \vert h \vert \leqslant c1\vert g \vert \leqslant \vert f \vert$. From this, we can transitively say $c5 \vert h \vert \leqslant \vert f \vert$. \\
\\
To prove $\vert f \vert \leqslant c6 \vert h \vert$, multiply $\vert g \vert \leqslant c4 \vert h \vert$ by c2 to get $c2 \vert g \vert \leqslant c2c4\vert h \vert$ \\
c2c4, the product of these two arbitrary numbers is another arbitrary number. Therefore, we can say c2c4 = c6. Substitute this into the inequality and it gives us $c2 \vert g \vert \leqslant c6\vert h \vert$. Using the other segmented inequality of $\vert f \vert \leqslant c2\vert g \vert$, we have $\vert f \vert \leqslant c2\vert g \vert \leqslant c6\vert h \vert$. From this, we can transitively say $\vert f \vert \leqslant c6\vert h \vert$. \\
\\
When $n \geqslant n1$, $c1 \vert g \vert \leqslant \vert f \vert$ \\
When $n \geqslant n2$, $c3 \vert h \vert \leqslant \vert g \vert$ \\
\\
When $n \geqslant n1$ and $n \geqslant n2$
\end{solution}

\goodbreak
\checklist
\end{document}

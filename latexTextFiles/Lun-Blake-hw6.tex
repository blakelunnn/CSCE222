\documentclass{article}
\usepackage{amsmath,amssymb,amsthm,latexsym,paralist}

\theoremstyle{definition}
\newtheorem{problem}{Problem}
\newtheorem*{solution}{Solution}
\newtheorem*{resources}{Resources}

\newcommand{\name}[2]{\noindent\textbf{Name: #1}\hfill \textbf{UIN: #2}}
\newcommand{\honor}{\noindent On my honor, as an Aggie, I have neither
  given nor received any unauthorized aid on any portion of the
  academic work included in this assignment. Furthermore, I have
  disclosed all resources (people, books, web sites, etc.) that have
  been used to prepare this homework. \\[2ex]
 \textbf{Electronic signature:} \underline{ \textbf{Blake Lun} } }
 
\newcommand{\checklist}{\noindent\textbf{Checklist:}
\begin{compactitem}[$\Box$] 
\item Did you type in your name and UIN? 
\item Did you disclose all resources that you have used? \\
(This includes all people, books, websites, etc.\ that you have consulted.)
\item Did you sign that you followed the Aggie Honor Code? 
\item Did you solve all problems? 
\item Did you submit the .tex and .pdf files of your homework to the correct link on Canvas? 
\end{compactitem}
}

\newcommand{\problemset}[1]{\begin{center}\textbf{Problem Set #1}\end{center}}
\newcommand{\duedate}[1]{\begin{quote}\textbf{Due dates:} Electronic
    submission of \textsl{yourLastName-yourFirstName-hw6.tex} and 
    \textsl{yourLastName-yourFirstName-hw6.pdf} files of this homework is due on
    \textbf{#1} on \texttt{https://canvas.tamu.edu}. You will see two separate links
    to turn in the .tex file and the .pdf file separately. Please do not archive or compress the files.  
    \textbf{If any of the two files are missing or unreadable, you will receive zero points for this
    homework.}\end{quote} }

\newcommand{\N}{\mathbf{N}}
\newcommand{\R}{\mathbf{R}}
\newcommand{\Z}{\mathbf{Z}}


\begin{document}
\vspace*{-20mm}
\begin{center}
{\large
CSCE 222 Discrete Structures for Computing -- Fall 2021\\[.5ex]
Hyunyoung Lee\\}
\end{center}
\problemset{6}
\duedate{Monday, 10/19/2021 11:59 p.m.}
\name{ Blake Lun }{ 131001178 }
\begin{resources} (All people, books, articles, web pages, etc.\ that
  have been consulted when producing your answers to this homework)
\end{resources}
\honor

\bigskip

\noindent
Total 100 points.

\bigskip

\noindent
The problems are from the lecture notes posted on Perusall.

\medskip

\noindent
All five questions are proof-by-induction questions.  For each proof, strictly adhere 
to the format and the style we discussed in class.  Especially in the inductive step, 
carefully explain each step-by-step in your own words!  You will not receive full credit 
if you don't explain the steps.

Grading rubrics for each induction proof are as follows: 20\% for the induction base
(clearly stating the base case and showing that the base case holds), 70\% for the
inductive step (correctly stating the induction hypothesis, clearly stating where
and how the induction hypothesis is used in the rest of the inductive step, clearly
and completely showing the derivation steps in the inductive step), and 10\% for the 
concluding remark of your proof. 

\medskip

\begin{problem} (20 points) Section 4.1, Exercise 4.3
\end{problem}
\begin{solution} 
NOTE: For question 1 and 2, the proof by induction method they used in the textbook follows this format. Initially I was using algebraic methods to show that P(n) $\to$ P(n + 1) but realized that the textbook had a much simpler method. If the goal was to show that they are equal algebraically, it didn't say so in the instructions. But if that is the case I understand that I'll have to have points taken off. I just hope if it is indeed the case that not too many points are taken off. \\
\\
Let P(n) be $(n(n + 1)(2n + 1)) / 6 = (k = 1, n)\Sigma k^2$ \\
Base Case: \\
P(n) holds for n = 1 since \\
$((1)((1) + 1)(2(1) + 1)) / 6 = (k = 1, 1)\Sigma k^2$ \\
$\to ((2)(3)) / 6 = (k = 1, 1)\Sigma k^2$ \\
$\to 1 = 1^2$ \\
\\
Induction Step: \\
Induction Hypothesis: Assume P(n) is true. Show that P(n) implies P(n + 1) or $(n(n + 1)(2n + 1)) / 6 = (k = 1, n)\Sigma k^2$ implies \\
$((n + 1)((n + 1) + 1)(2(n + 1) + 1)) / 6 = (k = 1, n + 1)\Sigma k^2$ holds for $n \geqslant 1$ \\
\\
P(n + 1) \\
$((n + 1)((n + 1) + 1)(2(n + 1) + 1)) / 6 = (n(n + 1)(2n + 1)) / 6 + (n + 1)^2$\\
$((n + 1)((n + 1) + 1)(2(n + 1) + 1)) / 6 = (k = 1, n)\Sigma k^2 + (n + 1)^2$\\
Right hand side of equation: $(k = 1, n)\Sigma k^2 + (n + 1)^2$ = $(k = 1, n + 1)\Sigma k^2$. This is because $(k = 1, n + 1)\Sigma k^2$ is equal to $1^2 + 2^2 + ... + n^2 + (n+1)^2$. The $1^2 + 2^2 + ... + n^2$ portion is equal to $(k = 1, n)\Sigma k^2$. So if we substitute that back in, the equation becomes $(k = 1, n)\Sigma k^2 + (n + 1)^2$ which is why they are equal to each other.\\
Therefore, $((n + 1)((n + 1) + 1)(2(n + 1) + 1)) / 6$ = $(k = 1, n + 1)\Sigma k^2$ which proves the implication. \\
Therefore, we can conclude by induction that P(n) holds for all $n \geqslant 1$, which proves the claim.
\end{solution}

\begin{problem} (20 points) Section 4.1, Exercise 4.4
\end{problem}
\begin{solution} 
Let P(n) be $(n^2(n + 1)^2) / 4 = (k = 1, n)\Sigma k^3$ \\
Base Case: \\
P(n) holds for n = 1 since: \\
$(1^2(1 + 1)^2) / 4 = (k = 1, 1)\Sigma k^3$ \\
$\to (2^2) / 4 = (k = 1, n)\Sigma k^3$ \\
$\to 4 / 4 = 1 = 1^3$ \\
\\
Induction Step: \\
Induction Hypothesis: Assume P(n) is true. Show that P(n) implies P(n + 1) or $(n^2(n + 1)^2) / 4 = (k = 1, n)\Sigma k^3$ implies $((n + 1)^2((n + 1) + 1)^2) / 4 = (k = 1, n + 1)\Sigma k^3$ holds for $n \geqslant 1$ \\
\\
P(n + 1) \\
$((n + 1)^2((n + 1) + 1)^2) / 4 = (n^2(n + 1)^2) / 4 + (n + 1)^3$\\
$((n + 1)^2((n + 1) + 1)^2) / 4 = (k = 1, n)\Sigma k^3 + (n + 1)^3$\\
Right hand side of equation: $(k = 1, n)\Sigma k^3 + (n + 1)^3$ = $(k = 1, n + 1)\Sigma k^3$. This is because $(k = 1, n + 1)\Sigma k^3$ is equal to $1^3 + 2^3 + ... + n^3 + (n+1)^3$. The $1^3 + 2^3 + ... + n^3$ portion is equal to $(k = 1, n)\Sigma k^3$. So if we substitute that back in, the equation becomes $(k = 1, n)\Sigma k^3 + (n + 1)^3$ which is why they are equal to each other.\\
Therefore, $((n + 1)^2((n + 1) + 1)^2) / 4 = (k = 1, n + 1)\Sigma k^3$ which proves the implication. \\
Therefore, we can conclude by induction that P(n) holds for all $n \geqslant 1$, which proves the claim.
\end{solution}

\begin{problem} (20 points) Section 4.1, Exercise 4.6
\end{problem}
\begin{solution} 
Let P(n) be $2^2^n - 1$ is divisible by 3 \\
Base Case: \\
P(n) holds for n = 1 since: \\
$(2^2^(^1^) - 1) = 3$ (which is an integer divisible by 3) \\
\\
Induction Step: \\
Induction Hypothesis: Assume P(n) is true. Show that P(n) implies P(n + 1) or $(2^2^n - 1)$ is divisible by 3 implies $(2^2^(^n^+^1^) - 1)$ is divisible by 3\\
\\
If we're assuming that P(n) is true, then: \\
$(2^2^n - 1) = 3x$ where x is some integer (it's multiplied by 3 to show it's divisible by 3)\\
$\to 2^2^n - 1= 3x$ \\
$\to 2^2^n = 3x + 1$ \\
\\
P(n + 1) \\
$(2^2^(^n^+^1^) - 1)$ \\
= $2^2^n^+^2 - 1$ \\
= $2^2^n * 2^2 - 1$ ; substitute $3x + 1$ for $2^2^n$\\
= $(3x + 1) * 4 - 1$\\
= $12x + 4 - 1$ \\
= $12x + 3$ \\
= $3(4x + 1)$ \\
$(2^2^(^n^+^1^) - 1)$ = $3(4x + 1)$ \\
The right side of the equation has a 3 outside of the parenthesis. This shows that $(2^2^(^n^+^1^) - 1)$ is divisible by 3 while under the assumption that x is an integer.
Therefore, we can conclude by induction that P(n) holds for all $n \geqslant 1$, which proves the claim.
\end{solution}

\begin{problem} (20 points) Section 4.3, Exercise 4.15
\end{problem}
\begin{solution} 
Let P(n) be $(k = 1, n)\Sigma f2k = f2 + f4 + ... + f2n = f(2n + 1) - 1$ \\
Base Case: \\
P(n) holds for n = 1 since: \\
f2(1) = f(2(1) + 1) - 1 \\
$\to f2 = f3 - 1 = 1$ \\
\\
Induction Step: \\
Induction Hypothesis: Assume P(n) is true. Show that P(n) implies P(n + 1) or $(k = 1, n)\Sigma f2k = f2 + f4 + ... + f2n = f(2n + 1) - 1$ implies \\
$(k = 1, n + 1)\Sigma f2k = f2 + f4 + ... + f2n + f2(n + 1)= f(2(n + 1) + 1) - 1$ \\
\\
P(n + 1) \\
$(k = 1, n + 1)\Sigma f2k = (k = 1, n)\Sigma f2k + f2(n + 1)$. We are able to say this since $(k = 1, n)\Sigma f2k$ is equal to the $f2 + f4 + ... + f2n$ portion of $(k = 1, n + 1)\Sigma f2k$. We then substitute $f(2n + 1) - 1$ with $(k = 1, n)\Sigma f2k$ taken from our P(n). This gives us: \\
$(k = 1, n + 1)\Sigma f2k = (f(2n + 1) - 1) + f(2n + 2)$. From here we are able to combine f(2n + 1) and f(2n + 2) into f(2n + 3. This is an identity that we can derive from these arbitrary numbers. Which would then give us: \\
$(k = 1, n + 1)\Sigma f2k = f(2n + 3) - 1$. If we look back at our equation from earlier; $(k = 1, n + 1)\Sigma f2k = f2 + f4 + ... + f2n + f2(n + 1)= f(2(n + 1) + 1) - 1$, this can be simplified to $(k = 1, n + 1)\Sigma f2k = f(2n + 3) - 1$. If we look back at the equation we obtained just a second ago, we see that they match and are the same thing. \\
Therefore, we can conclude by induction that P(n) holds for all $n \geqslant 1$, which proves the claim.
\end{solution}

\begin{problem} (20 points) Section 4.3, Exercise 4.17
\end{problem}
\begin{solution} 
Let P(n) be $f(n + 1)f(n - 1) - (fn)^2 = (-1)^n$ \\
Base Case: \\
P(n) holds for n = 2 since: \\
$f(2 + 1)f(2 - 1) - (f(2))^2 = (-1)^2$ \\
$f(3)f(1) - (f2)^2 = 1$ \\
$(2)(1) - (1)^2 = 1$ \\
$1 = 1$ \\
\\
Induction Step: \\
Let S(n) be $f(n + 1)f(n - 1) - (fn)^2$
Induction Hypothesis: Assume S(n) is true. Show that S(n) implies S(n + 1) or $f(n + 1)f(n - 1) - (fn)^2$ implies \\
$- f((n + 1) + 1)f((n + 1) - 1) + (f(n + 1))^2$ holds for $n \geqslant 2$. The signs are flipped since every number is the negative of the one before it.\\
For this question we will use the Fibonacci identity f(n - 1) = f(n + 1) - f(n) for all $n \geqslant 2$ \\
We are trying to show $f(n + 1)f(n - 1) - (fn)^2 \to f((n + 1) + 1)f((n + 1) - 1) - (f(n + 1)^2)$ \\
First simplify this to $f(n + 1)f(n - 1) - (fn)^2 \to -f(n + 2)f(n) + (f(n + 1))^2$ \\
Now looking at just the left hand side, let's get the two sides to match \\
$f(n + 1)f(n - 1) - (fn)^2$; multiply by -1 \\
$\to -f(n + 1)f(n - 1) + (fn)^2$; use the identity to substitute f(n - 1) \\
$\to -f(n + 1)(f(n + 1) - f(n)) + (fn)^2$; distribute -f(n + 1) \\
$\to -(f(n + 1)^2) + f(n + 1)(f(n) + (fn)^2$; factor out f(n) from $f(n + 1)(f(n)$ and $(fn)^2$ \\
$\to f(n)(f(n + 1) + f(n)) - (f(n + 1)^2)$; f(n + 1) + f(n) is the same as f(n + 2) \\
$\to f(n + 2)f(n) - (f(n + 1)^2)$ \\
We have arrived at $f(n + 2)f(n) - (f(n + 1)^2)$ which is the same as our original equation $- f((n + 1) + 1)f((n + 1) - 1) + (f(n + 1))^2$.
Therefore, we can conclude by induction that P(n) holds for all $n \geqslant 1$, which proves the claim.
\end{solution}

\goodbreak
\checklist
\end{document}

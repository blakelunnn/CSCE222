\documentclass{article}
\usepackage{amsmath,amssymb,amsthm,latexsym,paralist}

\theoremstyle{definition}
\newtheorem{problem}{Problem}
\newtheorem*{solution}{Solution}
\newtheorem*{resources}{Resources}

\newcommand{\name}[2]{\noindent\textbf{Name: #1}\hfill \textbf{UIN: #2}}
\newcommand{\honor}{\noindent On my honor, as an Aggie, I have neither
  given nor received any unauthorized aid on any portion of the
  academic work included in this assignment. Furthermore, I have
  disclosed all resources (people, books, web sites, etc.) that have
  been used to prepare this homework. \\[2ex]
 \textbf{Electronic signature:} \underline{ \textbf{Blake Lun} } }
 
\newcommand{\checklist}{\noindent\textbf{Checklist:}
\begin{compactitem}[$\Box$] 
\item Did you type in your name and UIN? 
\item Did you disclose all resources that you have used? \\
(This includes all people, books, websites, etc.\ that you have consulted.)
\item Did you sign that you followed the Aggie Honor Code? 
\item Did you solve all problems? 
\item Did you submit the .tex and .pdf files of your homework to the correct link on Canvas? 
\end{compactitem}
}

\newcommand{\problemset}[1]{\begin{center}\textbf{Problem Set #1}\end{center}}
\newcommand{\duedate}[1]{\begin{quote}\textbf{Due dates:} Electronic
    submission of \textsl{yourLastName-yourFirstName-hw5.tex} and 
    \textsl{yourLastName-yourFirstName-hw5.pdf} files of this homework is due on
    \textbf{#1} on \texttt{https://canvas.tamu.edu}. You will see two separate links
    to turn in the .tex file and the .pdf file separately. Please do not archive or compress the files.  
    \textbf{If any of the two files are missing or unreadable, you will receive zero points for this
    homework.}\end{quote} }

\newcommand{\N}{\mathbf{N}}
\newcommand{\R}{\mathbf{R}}
\newcommand{\Z}{\mathbf{Z}}


\begin{document}
\vspace*{-20mm}
\begin{center}
{\large
CSCE 222 Discrete Structures for Computing -- Fall 2021\\[.5ex]
Hyunyoung Lee\\}
\end{center}
\problemset{5}
\duedate{Monday, 10/11/2021 11:59 p.m.}
\name{ Blake Lun }{ 131001178 }
\begin{resources} (All people, books, articles, web pages, etc.\ that
  have been consulted when producing your answers to this homework)
\end{resources}
\honor

\bigskip

\noindent
Total 100 points.

\bigskip

\noindent
The problems are from the lecture notes posted on Perusall.

\medskip

\noindent
Always carefully explain your answers in your own words!

\medskip

\begin{problem} ($15+10=25$ points) Section 6.1, Exercise 6.2 (a) and (b).
[\textbf{Requirements:} For (a), you need to show carefully that the relation satisfies the
three properties, reflexivity, antisymmetry, and transitivity.  For (b), you need to
give a general condition for two elements $(a_1, a_2)$ and $(b_1, b_2)$ to be
incomparable, not some examples.]
\end{problem}
\begin{solution} 
The set is reflexive since every element is related to itself. There is a (1,1), (2,2), (3,3) ... (m,n) and there is only one of each of these. \\

\end{solution}

\begin{problem} (15 points) Determine the cover relation of the set of positive integers
from 1 to 10 ordered by divisibility $(\{1, \ldots, 10\},\, | \, )$.

\smallskip
\noindent
[Hint: Study Examples 6.13 and 6.17 in the lecture notes posted on Perusall.
The general definition of divisibility can be found in the first paragraph of page 63 
in Section 2.9 and in the problem statement of Exercise 6.1 on page 155.]

\smallskip
\noindent
[Notes/Rubrics: Cover relation is a relation, which is a set of ordered pairs, thus the 
ordered pairs must be enclosed in a set \{\,\} notation. If the set notation is missing, then 
points will be taken off. Each missing or superfluous ordered pair will also get points taken off.]
\end{problem}
\begin{solution} 
cover relation = \{(1, 2), (2, 4), (2, 6), (4, 8), (1, 3), (3, 6), (3, 9), (1, 5), (2, 10), (5, 10), (1, 7)\}
\end{solution}

\begin{problem} (30 points) Section 3.8, Exercise 3.60.
[Grading rubric: Giving a correct, concrete bijective function is worth ten points, and showing
that your function is injective and surjective worth ten points each.]
\end{problem}
\begin{solution} 
Two functions have the same cardinality if there is a bijective function from one set to the other. \\
Let f : $\N \in \Z$ \\
----------\{ x/2 when x is even\\
f(x) = \{ 0 when x = 0 \\
----------\{ -(x + 1) / 2 when x is odd \\
This function is both injective and surjective making it bijective. It is surjective because every integer is mapped to. If x is odd, the x is mapped to a negative number. If x is 1, f(x) is -1. If x is 3, f(x) is -2. If you continue this pattern all negative numbers have an x. If x is even, the x is mapped to a positive number. If x is 2, f(x) is 1. If x is 4, f(x) is 2. If you continue this pattern all positive numbers also have an x. Which leads us to the conclusion every Z has some A making this surjective. This function is also injective since every integer has a unique mapping. If we look at the pattern from the surjective example, when x is even, if x is 2, f(x) is 1. If x is 4, f(x) is 2. If x is 6, f(x) is 3. We can clearly see that we are incrementing through every even number of x and every positive integer has a unique mapping from x. The same can be said for when x is odd. If x is 1, f(x) is -1. If x is 3, f(x) is -2. If x is 5, f(x) is -3. Again, we are incrementing through all the odd values of x and every negative integer has a unique mapping of x. This allows us to come to the conclusion that the function is one-to-one, or every Z doesn't have many A's making it injective. Since we proved it is both injective and surjective, by default we can come to the conclusion that the function is bijective.
\end{solution}

\begin{problem} (5 points $\times$ 2 = 10 points) Section 7.1, Exercise 7.1 (a) and (d).
[\textbf{Requirements:} For (a), express your final answer in a format involving some power of 2.] 
\end{problem}
\begin{solution} 
a. ceiling: $ \lceil (2^3^2 - 1) / 8 \rceil \\
= \lceil (536870911.9) \rceil \\
= 536870912 \\
= 2^3^2 \\$
\\
floor: $\lfloor (2^3^2 - 1) / 8 \rfloor \\
= \lfloor (536870911.9) \rfloor \\
= 536870911 \\
= 2^3^2 - 1$ \\
\\
d. ceiling: $ \lceil \pi ^\pi \rceil \\
= \lceil (36.4621596072) \rceil \\
= 37$ \\
\\
floor: $\lfloor \pi ^\pi \rfloor \\
= \lfloor (36.4621596072) \rfloor \\
= 36$ \\
\end{solution}

\begin{problem} ($10+10=20$ points) Section 7.1, Exercise 7.2.
[\textbf{Requirements:} Use the definitions of $\lfloor \, \rfloor$ and $\lceil \, \rceil$ involving 
the inequalities, and carefully explain your algebraic reasoning. Using just some examples 
will hardly get you any points.]
\end{problem}
\begin{solution}
a. The floor function $\lfloor x \rfloor$ rounds the real number x down to the greatest integer that is smaller or equal to x. From that statement, we can get the right side of the inequality n $\leqslant$ x. It is an $\leqslant$ sign rather than a $<$ because if x is already an integer, then n would be equal to x. For example, if $n = \lfloor x \rfloor$ and $x = 3$, we have $\lfloor 3 \rfloor$ which is just $n = 3$. Since n is an integer, we need to cover an integers worth of a range of number on both sides of the inequality to cover all possible numbers. This gives us the left side of the inequality $x - 1 < n$. This time, the sign is $<$ rather than $\leqslant$ because we don't want to include the number an integer distance from x. We would be including a possible integer that should not be included in the range. For example, if $n = \lfloor x \rfloor$ and $x = 3$, we have $\lfloor 3 \rfloor$. Hypothetically, if the inequality was $x - 1 \leqslant n$ rather than $x - 1 < n$, we would be including the integer 2 in the range of possibilities. \\
\\
b. The ceiling function $\lceil x \rceil$ rounds the real number x up to the smallest integer that is greater or equal to x. From that statement, we can get the left side of the inequality x $\leqslant$ n. It is an $\leqslant$ sign rather than a $<$ because if x is already an integer, then n would be equal to x. For example, if $n = \lceil x \rceil$ and $x = 3$, we have $\lceil 3 \rceil$ which is just $n = 3$. Since n is an integer, we need to cover an integers worth of a range of number on both sides of the inequality to cover all possible numbers. This gives us the right side of the inequality $n < x + 1$. This time, the sign is $<$ rather than $\leqslant$ because we don't want to include the number an integer distance from x. We would be including a possible integer that should not be included in the range. For example, if $n = \lceil x \rceil$ and $x = 3$, we have $\lceil 3 \rceil$. Hypothetically, if the inequality was $n \leqslant x + 1$ rather than $n < x + 1$, we would be including the integer 4 in the range of possibilities. \\
\end{solution}

\goodbreak
\checklist
\end{document}

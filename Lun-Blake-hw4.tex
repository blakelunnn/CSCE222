\documentclass{article}
\usepackage{amsmath,amssymb,amsthm,latexsym,paralist}

\theoremstyle{definition}
\newtheorem{problem}{Problem}
\newtheorem*{solution}{Solution}
\newtheorem*{resources}{Resources}

\newcommand{\name}[2]{\noindent\textbf{Name: #1}\hfill \textbf{UIN: #2}}
\newcommand{\honor}{\noindent On my honor, as an Aggie, I have neither
  given nor received any unauthorized aid on any portion of the
  academic work included in this assignment. Furthermore, I have
  disclosed all resources (people, books, web sites, etc.) that have
  been used to prepare this homework. \\[2ex]
 \textbf{Electronic signature:} \underline{ \textbf{Blake Lun} } }
 
\newcommand{\checklist}{\noindent\textbf{Checklist:}
\begin{compactitem}[$\Box$] 
\item Did you type in your name and UIN? 
\item Did you disclose all resources that you have used? \\
(This includes all people, books, websites, etc.\ that you have consulted.)
\item Did you sign that you followed the Aggie Honor Code? 
\item Did you solve all problems? 
\item Did you submit the .tex and .pdf files of your homework to the correct link on Canvas? 
\end{compactitem}
}

\newcommand{\problemset}[1]{\begin{center}\textbf{Problem Set #1}\end{center}}
\newcommand{\duedate}[1]{\begin{quote}\textbf{Due dates:} Electronic
    submission of \textsl{yourLastName-yourFirstName-hw4.tex} and 
    \textsl{yourLastName-yourFirstName-hw4.pdf} files of this homework is due on
    \textbf{#1} on \texttt{https://canvas.tamu.edu}. You will see two separate links
    to turn in the .tex file and the .pdf file separately. Please do not archive or compress the files.  
    \textbf{If any of the two files are missing or unreadable, you will receive zero points for this
    homework.}\end{quote} }

\newcommand{\N}{\mathbf{N}}
\newcommand{\R}{\mathbf{R}}
\newcommand{\Z}{\mathbf{Z}}


\begin{document}
\vspace*{-20mm}
\begin{center}
{\large
CSCE 222 Discrete Structures for Computing -- Fall 2021\\[.5ex]
Hyunyoung Lee\\}
\end{center}
\problemset{4}
\duedate{Friday, 10/1/2021 11:59 p.m.}
\name{ Blake Lun }{ 131001178 }
\begin{resources} (All people, books, articles, web pages, etc.\ that
  have been consulted when producing your answers to this homework)
\end{resources}
\honor

\bigskip

\noindent
Total 105 points.

\bigskip

\noindent
The problems are from the lecture notes posted on Perusall.

\medskip

\begin{problem} (15 points) Section 3.1, Exercise 3.9.  First, describe the set
as a geometric shape, and then list the elements of the set $E$ in curly braces.  
\end{problem}
\begin{solution} 
Note: 0 is not a nonnegative integer but for this question I assumed 0 could be used otherwise the set would be empty. \\
The geometric shape is an ellipse. \\
$E = \{ (0,0), (0, 1), (1, 0), (2, 0) \}$
\end{solution}

\begin{problem} (15 points) Section 3.1, Exercise 3.10.  First, list the elements of 
the set $S$ in curly braces, and then give the power set $P(S)$.
\end{problem}
\begin{solution} 
S = \{-1, 0, 1\}  \\
P(S) = \{ \varnothing, \{-1\}, \{0\}, \{1\}, \{-1, 0\}, \{0, 1\}, \{-1, 1\}, \{-1, 0, 1\}\}
\end{solution}

\begin{problem} (5 points $\times$ 2 = 10 points) Section 3.2, Exercise 3.17
\end{problem}
\begin{solution} 
\ \\
\textbf{(a)}
Let $S$ be the family $S = \{ A,B,C \}$. S has the intersection $\cap S = C$. The numbers for this intersection is restricted only by set C (the powers of 6). Any number that is a power of 6 is even since powers of 6 are numbers that are comprised of even numbers multiplied by another even number. This means all powers of 6 are even numbers. So $A \cap C$ = $C$. So the formula $A \cap B \cap C$ can be rewritten as $B \cap A \cap C$ which can be rewritten again as $B \cap C$. Since 3 is a multiple of 6, any number that can be divided by 6 can also be divided by 3. This means that all numbers that are powers of 6 are multiples of 3. This however does not apply to $0^6$ and $1^6$. Therefore, $\cap S = \{ x \in C | x \geqslant 2^6\}$
\ \\[2ex]
\textbf{(b)}
Let $S$ be the family $S = \{ A,B \}$. We are describing $A \cup B$. This would include all even integers (multiples of 2) and all multiples of 3. If you look at the numbers that are left after taking those possibilities out, you're left with [1, 5, 7, 11, 13, 17, 19...]. If you didn't notice, not including the number 1 and 2, these are all primes numbers. So the $\cup S$ can be described as all primes numbers not including 2.
\end{solution}

\begin{problem} (30 points) Section 3.3, Exercise 3.26 
[Requirements: Use the definition of set difference involving complement. It will be easier 
if you start from the right-hand side of the equality.  Carefully derive step-by-step using 
de Morgan's law for complements and the properties involving $\cup$ and $\cap$, 
finally reaching the left-hand side.]
\end{problem}
\begin{solution} 
Prove $A \cap (B - C) = (A \cap B) - (A \cap C)$ \\
$(A \cap B) - (A \cap C)$ = $(A \cap B) \cap (A \cap C)^C$ since $A - B = A \cap B^C$ \\
$(A \cap B) - (A \cap C)$ = $(A \cap B) \cap (A^C \cap C^C)$ by de Morgan's law \\
$(A \cap B) - (A \cap C)$ = $((A \cap B) \cap A^C) \cup ((A \cap B) \cap C^C)$ by distributive law \\
$(A \cap B) - (A \cap C)$ = $((A \cap A^C) \cap B) \cup (A \cap (B \cap C^C)$ by associative law \\
$(A \cap B) - (A \cap C)$ = $(\varnothing \cap B) \cup (A \cap (B - C)$ since $A \cap A^C = \varnothing$ and $A \cap B^C = A - B$ \\
$(A \cap B) - (A \cap C)$ = $\varnothing \cup (A \cap (B - C))$ \\
$(A \cap B) - (A \cap C)$ = $A \cap (B - C)$ since $\varnothing \cup A = A$

\end{solution}

\begin{problem} (20 points) Section 3.4, Exercise 3.33
[Requirements: Show that the left-hand side of the equality is a subset of the right-hand 
side, and vice versa.  Show your reasoning using the definitions of union ($\cup$), 
Cartesian product ($\times$), and subset ($\subseteq$).]
\end{problem}
\begin{solution} 
$(A \cup B) \times C = \{(a, b)$ in $P(P(((A \cup B) \times C)) | a \in (A \cup B)$ and $b \in C$ \\
This statement says that the set a subset within $(A \cup B) \times C$ will be the set that we are looking for. For the second portion
\end{solution}

\begin{problem} (15 points) Section 5.1, Exercise 5.4
[Hint: A relation is an equivalence relation if and only if it is reflexive, symmetric, and
transitive.  Thus, you need to show that the given relation satisfies the three properties.]
\end{problem}
\begin{solution} 
Reflexive: \\
$x/x = 2^k, k = 0, x \in N $\\
Therefore, x $\sim$ x making it reflexive. \\
\\
Symmetric: \\
Let $x \sim y; x, y \in N$ \\
$x / y = 2^k, k$ is some integer. Switching the places of the variables in the fraction gives $y / x = 2^-k$. Since k is an integer in the first equation and -k is also an integer (it's the same number just the negative version), it is symmetric since $x \sim y$ and $y \sim x$. \\
\\
Transitive: \\
Let $x \sim y$ and $y \sim z$ where $x, y, z \in N$ \\
Let $x / y = 2^k^1$ and Let $y / z = 2^k^2$ where k1 and k2 are integers. \\
If you multiply the two different sides respectively, you get $(x / y) * (y / z) = 2^(^k^1 ^+ ^k^2^)$. Since both are equal to a base of 2, when multiplying them together, you can keep the base and add the exponents to be the new exponent. Now we can cancel out y since it's the denominator for x and the numerator for z to get $x / z = 2^k^3$. \\
Since k1 and k2 are integers, k3 must also be an integer.
Therefore we can say since $x \sim y$ and $y \sim z$, $x \sim z$ by transitive property.\\
\\
Therefore showing it is an equivalence relation because its reflexive, symmetric, and transitive.

\end{solution}

\goodbreak
\checklist
\end{document}
